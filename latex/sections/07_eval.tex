\chapter{Final Evaluation}

    \section{Success Criteria}

        Success criteria have been evaluated at the end of each cycle. 
        See the end of each cycle for success criteria. 


    \section{Stakeholder Feedback}

        Stakeholder feedback is very important, as they represent the demographic that is most likely to be using the final product.

        % Talk about consistent user interface
        % Talk about more components
        % Talk about Open Source

        One concern for stakeholders was an inconsistent user interface.
        Having a consistent user interface is important, as if a user finds software frustrating to use then they are less likely to use it in the future. 
        The user interface will be designed in a more structured way, and could involve trialling many different designs in order to see which ones users find the most intuitive. 

        Another concern of stakeholders was the lack of variety of components. 
        One concept that could help with this is the idea of plug-ins or packages.
        These could be user-developed (with some basic ones coming pre-installed by default), containing components that fit specific use cases, e.g. someone could make a plug-in that has common components used in audio engineering, and the user would be able to decide which plug-ins to include with their project.
        This would also prevent the user interface from getting too cluttered with components the user isn't currently using.
        This could also encourage more people to get into programming (in context of developing their own plug-ins), as well as establishing an online community around the product, possibly using a service such as Discord. 

        However, the key focus on future development should be on achieving working circuits in parallel, as that would greatly expand user possibilities and functionality of the software. 
