\chapter{Cycle 1: Installation}
\graphicspath{{figures}}


Below is the hierarchy diagram for this cycle. Note that here `program' is used to refer to the entire program, and not just the installer covered in this cycle.

% for:hierarchy_diagram_c1
% Cycle 1 Hierarchy diagram
\begin{figure}[!ht]
    \centering
    \begin{forest}
        for tree={
            align=center,
        edge+={thick, -{Stealth[]}},
        l sep'+=10pt,
        fork sep'=10pt,
        },
        forked edges,
        if level=0{
            inner xsep=0pt,
            tikz={\draw [thick] (.children first) -- (.children last);}
            }{},
            [Installation
                [File management
                    [Hosting files]
                    [Updating files]
                ]
                [Installer
                    [Checking\\permissions
                        [Admin\\permissions]
                        [Directory\\validation]
                    ]
                    [Program\\configuration]
                    [Ease of use
                        [Add desktop\\shortcut]
                        [Uninstaller\\program]
                    ]
                ]
            ]
    \end{forest}
    \caption{Cycle 1 hierarchy diagram.}
    \label{for:hierarchy_diagram_c1}
\end{figure}

\section{Brief Outline}
    In this cycle I will create a way to install the program onto the user's computer using Python. 
    This will also include an uninstallation program which will allow the user to safely uninstall the program if they wish to do so.

    \subsection{Success Criteria}
        See \autoref{tbl:succ_crit_c1}.
        
\section{Design}
    % Need to do UI and testing here, as well as talk more about how this is going to be implemented

    \subsection{User Interface}
        User interface (UI) windows will be implemented as classes that then have individual elements such as buttons, entries, and labels implemented as attributes of the class that can be edited as needed. 
        Buttons will call methods in the class as their functions, allowing them to refer to other UI elements in the same window easily. See \autoref{fig:install_gui_class_diagram_c1}.

        % fig:install_gui_class_diagram_c1
        % InstallGUI Class Diagram FIG
        \begin{figure}[!ht]
            \centering
            \begin{tikzpicture}
                \begin{class}[text width=5cm]{InstallGUI}{0,0}
                    \attribute{btn\_cancel : Button}
                    \attribute{btn\_install : Button}
                    \attribute{btn\_browse : Button}
                    \attribute{ent\_dir : Entry}
                    \attribute{lbl\_welcome : Label}
                    \attribute{filepath : String}

                    \operation{btn\_cancel\_command}
                    \operation{btn\_install\_command}
                    \operation{btn\_browse\_command}
                \end{class}
            \end{tikzpicture}
            \caption{InstallGUI class diagram}
            \label{fig:install_gui_class_diagram_c1}
        \end{figure}

        In \autoref{fig:installer_gui_design_c1}, I have made a design for the graphical user interface of the installer part of this cycle. 
        The user will be able to enter a directory on their computer in the white entry box, and then press the "Install" button which will then validate this input as well as download the files to the specified folder. 
        The cancel button will quit the installer and no further processing will occur.

        % fig:installer_gui_design_c1
        % design for the installer gui 
        \begin{figure}[!ht]
            \centering
            \includegraphics[width=.7\textwidth]{s03/design/installer_ui_design_mock.png}
            \caption{UI Design}
            \label{fig:installer_gui_design_c1}
        \end{figure}

        I chose the colours that I did for the button, as they were suggested to have high contrast with the black text on them, see \autoref{fig:high_contrast_ppt_pf_c1}. 
        This aids accessibility, making it easier to see the text on each button. 
        I believe that the text on each button also sufficiently describes their function, leaving no room for ambiguity, and the user can be confident in what each press will do. 

        % fig:high_contrast_ppt_pf_c1
        % Screenshot from ppt showing the "high-contrast only" toggle
        \begin{figure}[!ht]
            \centering
            \includegraphics[width=.7\textwidth]{s03/design/high_contrast_ppt_pf.png}
            \caption{High contrast colour suggestions by PowerPoint}
            \label{fig:high_contrast_ppt_pf_c1}
        \end{figure}

        I want to be able to communicate to the user if an error occurs, with a special case for if the user inputs an invalid directory. 
        I used text coloured red, which has negative connotations, that was also marked as high contrast, as well as messages that told the user that the input was invalid, or a message saying that an unknown error occurred, see \autoref{fig:installer_ui_design_errors_c1}. 
        I don't think that other errors will need to be treated specially, as any that occur would be with the downloading process and out of the user's control.

        % fig:installer_ui_design_errors_c1
        %       fig:installer_ui_design_directory_error
        %       fig:installer_ui_design_other_error
        % Showing how errors will be handled by the UI
        \begin{figure}[!ht]
            \begin{subfigure}{.5\textwidth}
                \centering
                \includegraphics[scale=0.5]{s03/design/installer_ui_design_error_1.png}
                \caption{Invalid directory error}
                \label{fig:installer_ui_design_directory_error}
            \end{subfigure}%
            \begin{subfigure}{.5\textwidth}
                \centering
                \includegraphics[scale=0.5]{s03/design/installer_ui_design_unknown_error.png}
                \caption{Other unknown error}
                \label{fig:installer_ui_design_other_error}
            \end{subfigure}%
            \caption{Design for communicating errors to the user}
            \label{fig:installer_ui_design_errors_c1}
        \end{figure}

        In the event of a successful installation, I would also like to notify the user. 
        To do this I used the word "Success!" coloured in high contrast green, which is a colour with positive connotations, shown in \autoref{fig:install_ui_successful_design_c1}.

        % fig:install_ui_successful_design_c1
        % installation ui after a successful installation
        \begin{figure}[!ht]
            \centering
            \includegraphics[width=.7\textwidth]{s03/design/installer_ui_success.png}
            \caption{Message after successful installation}
            \label{fig:install_ui_successful_design_c1}
        \end{figure}


    \subsection{Data Structures}

        The user may not always have permissions to save into/export into the directory they have selected. 
        When doing file management using for example the file explorer on Windows 10, the operating system will handle telling the user that they have invalid permissions. 
        However, when attempting operations on files with invalid permissions through a programming language such as Python, the error will be raised in the program. 
        Due to this I have included a \verb|permissions| variable that checks whether the program has permissions to write to a given directory to prevent this. 
        See \autoref{tbl:data_structs_c1} for a table of data structures.

    \subsection{Algorithms}
        I decided to plan some of my key algorithms for this cycle, namely the algorithms responsible for exporting files, initialising a project, and installing the program.

        % flow:flowchart_design_c1
        % flowchart for exporting a project as a .zip file
        \begin{figure}[!ht]
            \centering
            \begin{tikzpicture}[node distance=2cm]
                \node (start) [startstop] {Start};
                \node (in1) [io, below of=start, align=center] {Get destination\\directory};
                \node (dec1) [decision, below of=in1, align=center, yshift=-1.2cm] {User has\\permissions?};
                \node (exc1) [startstop, left of=dec1, xshift=-2.2cm] {Raise exception};
                \node (pro1) [process, below of=dec1, align=center, yshift=-1.2cm] {Move zip archive\\to destination};
                \node (end) [startstop, below of=pro1] {End};
                \draw [arrow] (start) -- (in1);
                \draw [arrow] (in1) -- (dec1);
                \draw [arrow] (dec1) -- node[anchor=north] {No} (exc1);
                \draw [arrow] (dec1) -- node[anchor=east] {Yes} (pro1);
                \draw [arrow] (pro1) -- (end);
            \end{tikzpicture}
            \caption{Flowchart for exporting a project}
            \label{flow:flowchart_design_c1}
        \end{figure}
        

        % pc:copy_to_folder_ps_c1
        % Initialise project PC
        \begin{figure}[!ht]
            \begin{minted}[linenos]{python}
import os
source = "Source directory here"
procedure init_files(target)
    if target.permissions == False:
        raise PermissionError
    files = os.listdir(source)
    for f in files:
        f.copy(target)
endprocedure
            \end{minted}
            \caption{Pseudocode to copy initial project files to a folder}
            \label{pc:copy_to_folder_ps_c1}
        \end{figure}


        The file running the procedure above will need to be in the same location as the folder it is copying from. 
        To make the file easier to access, the program could make a shortcut to this file and add it to the desktop for the user.
        

        % pc:install_software_ps_c1
        % Install main program PC
        \begin{figure}[!ht]
            \begin{minted}[linenos]{python}
url = https://github.com/nathan-tat/cs_nea_2025/tree/main/code/software
procedure install_software(target)
    if target.permissions == False:
        raise PermissionError

    download_from_github(url, target)
endprocedure
            \end{minted}
            \caption{Pseudocode to install the main program}
            \label{pc:install_software_ps_c1}
        \end{figure}


        The installer will copy files from a GitHub repository, where all the files necessary for the program to run will be stored, into a folder specified by the user on the user's computer. 
        This folder will then be referred to as \verb|source|, see \autoref{tbl:data_structs_c1}. 
        This will hopefully be done through the use of an external library.

        Having multiple directories with the program downloaded or having broken installations could cause undesired outcomes. 
        Due to this, I would like the user to use the "Close" button in order to terminate the program, and so I will include a check for this. 
        A Python file will not necessarily terminate after a GUI window is closed, so it is still possible to do some processing after this. 
        In the method written for the "Close" button, I will have a variable \verb|safety| set to \verb|True| and then check its value afterwards, alerting the user if this condition isn't met and possibly giving them some advice on how they can go forwards to guarantee that everything will work later on, see \autoref{fig:installer_ui_design_unsafe_exit_c1}.

        % fig:installer_ui_design_unsafe_exit_c1
        % Little design for when the user doesn't exit the program safely. 
        \begin{figure}[!ht]
            \centering
            \includegraphics[width=.7\textwidth]{s03/design/installer_ui_design_unsafe_exit.png}
            \caption{Unsafe termination of the program alert}
            \label{fig:installer_ui_design_unsafe_exit_c1}
        \end{figure}



\section{Implementation}
    % I need to ask if I'm doing a diary style implementation or the other one cause i have no clue
    % I find it easer to write some code, insert a screenshot of it working/not working, and then explaining why I did what I did/what I did to fix it

    % The installer will copy files from a specified repository on GitHub, which will store all the necessary files for the program to run, into a folder specified by the user, which will then be referred to by the program as the source folder. 


    % In \autoref{sc:if_name_main_c1}, there is code that is executed when the \verb|installer.py| file is run. This includes a check for whether the program is currently being run in administrator mode, shown in \autoref{sc:admin_checker}. This differs slightly from the design in \autoref{tbl:data_structs_c1}, where there was a variable defined to do the same thing. I found this implementation more readable and understandable, as using a function that returns a boolean value emphasises the fact that I want to check the permissions at a single moment in time. The code continues to start the UI so that the user can interact with the program. The code shown in \autoref{sc:admin_checker} is from StackOverflow. 
    % I am going to work on proper references at some point for this document


    When the Python interpreter reads a source file, it sets the variable \verb|__name__| to \verb|__main__| if the program is being run as the main program, or as the filename if the program is being imported.
    In order to avoid problems such as the installer being run unintentionally, I have decided to check the value of this variable to make certain that no code is executed in the case that the file is imported somewhere, see \autoref{sc:if_name_main_c1}.
    In this beginning if-statement, there is also a check for whether the program is being run using elevated privileges or not, using the function shown in \autoref{sc:admin_checker}. 
    This prevents the user from encountering errors when trying to install to a location that they don't have permission to.
    This part of the program is also when the GUI element is run from.


    % fig:installer_gui_ss1_c1
    % Installer GUI evidence FIG
    \begin{figure}[!ht]
        \centering
        \includegraphics[width=.7\textwidth]{s03/implement/installer_gui_pf1.png}
        \caption{Installer in action}
        \label{fig:installer_gui_ss1_c1}
    \end{figure}


    While implementing the UI elements, I discovered that `Tkinter', the Python library I decided to make use of for the UI, allowed for opening a pop-up dialogue in which the user could select a folder, as demonstrated in \autoref{fig:installer_gui_ss1_c1}.
    This avoided the need to validate the user's entry as well as the need for a user text input at all, since the user can only select directories that exist on their computer. 
    This also lead to the abandoning of a default install directory being entered for the user, which was initially introduced due to how awkward it would've been to input a directory.

    % fig:requirements-txt-c1
    % Requirements.txt file example
    \begin{figure}[!ht]
        \centering
        \includegraphics[width=.7\textwidth]{s03/implement/requirements_file.png}
        \caption{Requirements file}
        \label{fig:requirements-txt-c1}
    \end{figure}

    In the function tied to the `Install' button, see \autoref{sc:install_btn_sc_c1}, there are two functions called.
    The \verb|install_requirements| function installs all the necessary Python modules onto the user's device referencing a text file that contains the names and version numbers of said modules, shown in \autoref{fig:requirements-txt-c1}. 
    At first, I had some trouble with this function as I wasn't calling the command with the correct flags, shown in \autoref{sc:install_reqs_fail1_c1}. 
    The \verb|download_from_github| function copies the files from a folder in a GitHub repository. 
    % The former of these makes sure that the user has all the necessary Python libraries installed on their device, so that the program can function properly. 
    % Originally, I was planning to completely make use of a Python library that would have allowed me to interact with the GitHub API, but I could not find one that would do that for me and work, instead I managed to find code that would do the same thing, which is shown in \autoref{sc:dl_from_github_stolen_c1}, courtesy of GitHub user \href{https://github.com/Nordgaren/Github-Folder-Downloader/tree/master}{Nordgaren}.


    % As previously mentioned, I want to use a function in order to make sure that the relevant Python libraries are installed, which is shown in \autoref{sc:install_reqs_fail1_c1}. 
    % When this function is called in the program, see \autoref{sc:install_btn_sc_c1} line 22, the \verb|REQ| constant from is always passed as the argument, which links to the \verb|requirements.txt| file in the GitHub repository. However, this feature doesn't work as intended, see \autoref{fig:install_reqs_fail_pf_c1}, so it is commented out in the current version.


    % fig:install_reqs_fail_pf_c1
    % cmd ss proof to show that it failed
    \begin{figure}[!ht]
        \centering
        \includegraphics[width=.7\textwidth]{s03/implement/installer_req_fail1.png}
        \caption{Failure of install requirements function}
        \label{fig:install_reqs_fail_pf_c1}
    \end{figure}
    
    % fig:install-req-success
    % Success of install req function
    \begin{figure}[!ht]
        \centering
        \includegraphics[width=.7\textwidth]{s03/implement/install_req_success.png}
        \caption{Success of install requirements function}
        \label{fig:install-req-success}
    \end{figure}

    % I need to test it
    I then decided to implement the function in order to create a desktop shortcut, shown in \autoref{sc:create_shortcut_c1}. 
    For this I also made the change of adding a checkbox to the user interface that would allow the user to decide whether to add a desktop shortcut, shown in \autoref{fig:checkbox-installer-gui}. 


    % fig:checkbox-installer-gui
    % Installer GUI with checkbox
    \begin{figure}[!ht]
        \centering
        \includegraphics[width=.7\textwidth]{s03/implement/new_installer_gui.png}
        \caption{Installer GUI with checkbox}
        \label{fig:checkbox-installer-gui}
    \end{figure}

    Initially this feature didn't work, shown in \autoref{fig:shortcut-fail-c1}, but as after some tweaking to the file path, it worked, as shown in 

    % fig:shortcut-fail-c1
    % Shortcut failure
    \begin{figure}[!ht]
        \centering
        \includegraphics[width=.7\textwidth]{s03/implement/shortcut_make_fail_01.png}
        \caption{Shortcut not working}
        \label{fig:shortcut-fail-c1}
    \end{figure}

    % fig:
    % 
    \begin{figure}[!ht]
        \centering
        \begin{subfigure}{.4\textwidth}
            \centering
            \includegraphics[width=.9\textwidth]{s03/implement/shortcut-working-proof-01.png}
            \caption{The code in the file}
            \label{fig:sc-working-proof-1-c1}
        \end{subfigure}%
        \begin{subfigure}{.4\textwidth}
            \centering
            \includegraphics[width=.9\textwidth]{s03/implement/shortcut-working-proof-03.png}
            \caption{Code executed from the shortcut}
            \label{fig:}
        \end{subfigure}%
        \caption{Proof of shortcut working}
    \end{figure}


\section{Testing}

    See \autoref{tbl:test_data_before_c1} for the testing table for this cycle.

    % Apparently I need to include the Test ID whenever i reference them in the text and also conclude whether tests were successful or not

    % I actually need to do the tests in the first place 



\section{Evaluation}
    In this cycle, I have managed to reach all of my success criteria listed in \autoref{tbl:succ_crit_c1} except for being able to uninstall the program with the same file. 
    I have decided to move the development of this function to a later cycle, after the main program has been developed as I need a clearer view of how the program file structure will look like. 
