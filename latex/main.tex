\documentclass[11pt]{article}

\usepackage{array}
\usepackage{caption}
\usepackage{fancyhdr}
\usepackage[a4paper, margin=1in]{geometry}
\usepackage[final]{graphicx}
\usepackage[hidelinks]{hyperref}
\usepackage[newfloat]{minted}
\usepackage{xcolor}
\usepackage{subcaption}
\usepackage{listings,cleveref}
% \graphicspath{{./figures/}}

\definecolor{LightGray}{gray}{0.9}


% \newenvironment{code}{\captionsetup{type=listing}}{}
% \SetupFloatingEnvironment{listing}{name=Source Code}


\title{Computer Science NEA 2025}
\author{Nathan Tatkowski}
\date{\today}

\begin{document}
    \pagestyle{fancy}
    \setlength{\headheight}{13.6pt}
    \maketitle
    \pagebreak
    \tableofcontents
    \pagebreak
    \fancyhead[L]{Nathan Tatkowski}


    \section{Analysis}
        \subsection{Identification of Problem}
            Often times during research it is important and very helpful to be able to visualise the events that are being analysed. When working with models in two dimensions, it is easy enough to be able to draw out an accurate diagram, even if it may be tedious. However, 2D models are nowhere near as applicable or useful as considering events in three dimensions, stemming from the fact that the world we live in is three-dimensional. Something that would aid intuition and help in problem-solving would be a way to have events modelled quickly, accurately, and clearly, given a set of initial conditions.
        \subsection{Identification of why this problem is solvable by computational methods}
        \subsection{Description of the Current System}
        \subsection{Stakeholders}
        \subsection{Identification of User Needs and Acceptable Limitations}
        \subsection{Existing Solutions}
        \subsection{Success Criteria of the Proposed System}
        \subsection{Data Source(s)}
        \subsection{Volumetrics - Data Volumes}
        \subsection{Analysis Data Dictionary}
        \subsection{Data Flow Diagrams for Existing and Proposed System}
        \subsection{Justification of Chosen Solution}
        \subsection{Hardware and Software}
        \subsection{Entity-Relationship Models}
        \subsection{Identification of Objects and Object Analysis Diagrams}

    \section{Design}
        \subsection{Overall System Design}
        \subsection{Description of Modular Structure of System}
        \subsection{Definition of Data Requirements}
        \subsection{Identification of Appropriate Storage Media}
        \subsection{Identification of Processes and Suitable Algorithms for Data Transformation and Completion of the Solution}
        \subsection{Sample of Algorithms}
        \subsection{User Interface Design Rationale and Usability Features}
        \subsection{Security and Integrity of Data}
        \subsection{System Security (Access Control)}
        \subsection{Overall Test Strategy}

    \section{Implementation}
        \subsection{Annotated Listing of the Program(s)}
        \subsection{Annotated "Design Views" showing details of application-generated forms, reports, queries, buttons, cross tabulations, etc.}
        \subsection{Procedure and Variable List}
        \subsection{Testing to inform development - Testing at each stage}
        \subsection{Re-Testing}

    % Join with previous section 
    \section{Testing}
        \subsection{Test Plan}
        \subsection{Test Data}
        \subsection{Areas to Test}
        \subsection{Tables}
        \subsection{Justification of Data Selection}
        \subsection{Evidence of Testing}
        \subsection{Program Listing}
    
    \section{Evaluation}
        \subsection{Comparison of Performance of your System against Success Criteria}
        \subsection{Analysis of User Feedback}
        \subsection{Evaluation of Usability Features}
        \subsection{Maintenance Issues and Limitations of the Product}
        \subsection{Improvements and Possible Extensions}
        

\end{document}